\chapter{Conclusion}\label{ch:conclusion}
\setlength{\epigraphwidth}{0.90\textwidth}
\epigraph{``We are all shaped by the tools we use, in particular: the formalisms we use shape our thinking habits, for better or for worse, and that means that we have to be very careful in the choice of what we learn and teach, for unlearning is not really possible.''}{\begin{flushright}--Edsger W. \citet{dijkstra2000answers}, \href{https://www.cs.utexas.edu/~EWD/transcriptions/EWD13xx/EWD1305.html}{\textit{Answers to questions from students of Software Engineering}}\end{flushright}}

In this work, we explored four different programming tools from software engineering for the development of intelligent systems, broadly addressing cognitive complexity arising in four phases of Royce's Waterfall method~\autoref{fig:waterfall_model}. These tools have varying degrees of practicality, from highly theoretical (e.g. adversarial testing of differentiable programs ~\autoref{ch:difftest}) to more pragmatic (e.g. containerization ~\autoref{ch:ducker}). In each chapter, we provide some motivating examples and use cases which demonstrate key deficiencies in state-of-the-art programming tools for intelligent systems and propose candidate solutions which address a few of those shortcomings. While we certainly hope that intelligent system programmers (e.g. roboticists and machine learning practitioners) may derive some value from the tools themselves, our intention is to be \textit{illustrative} rather than \textit{prescriptive}.

In building tools and validating their effectiveness for toy applications, we hope that middleware and tools developers will carefully consider the cognitive complexity which software tools can introduce and the importance of notational and denotational design. Notation forces authors to think carefully about their abstractions, prevents logical errors, and allows the reader to more easily understand the implications. We hope that the programming tools illustrated in this thesis will inspire developers to re-imagine the potential for computer-aided programming in the design of intelligent systems.

By complementing the cognitive abilities of human programmers -- who excel at creative problem solving and high-level abstract reasoning -- with the low-level symbolic processing capabilities of programming tools, we can accelerate the design~\autoref{ch:hatchery}, development~\autoref{ch:kotlingrad}, validation~\autoref{ch:difftest} and deployment~\autoref{ch:ducker} of intelligent systems in real-world applications. This process, we argue, deserves more specific tools than general-purpose programming due to the opportunities and challenges which intelligent systems present and the unique interplay between human and machine intelligence.

As we start to engineer autonomous systems which take increasingly human decisions, programmers will play a critical role in shaping the behavior and dynamics of these systems. In order to build trustworthy autonomous systems, tools which enable humans to reason about the behavior of autonomous systems are essential~\citep{famelis2012partial}. Doing so requires us to actively rethink the programming model in machine learning to incorporate human knowledge, e.g. using differentiable programming and type theory (\autoref{ch:kotlingrad}) building tools which provide automated reasoning and visualization capabilities such as customized run and debugging assistance (e.g. \autoref{ch:hatchery}). Ensuring software artifacts are reproducible will require sound build systems and best practices for reproducible software installation and configuration (e.g. \autoref{ch:ducker}).

Traditional software engineering prescribes a rigorous process model and testing methodology~\autoref{fig:waterfall_model} which has guided generations of software projects. To become a true engineering discipline, the machine learning community will need to re-imagine this paradigm for systems which continously adapt to their environment. Intelligent systems are trained on \textit{objective functions}, which are typically one- or low-dimensional functions for measuring the performance of a system, typically outputting a scalar value known as \textit{error} or \textit{loss}. In practice, we care about a multiobjective set of criteria~\citep{censi2015mathematical}, including energy efficiency~\citep{paull2010novel}, memory~\citep{memory2013mitliagkas}, re/usability~\citep{breuleux2017automatic, deleu2019torchmeta}, predictability~\citep{turner2017well}, latency~\citep{ravanelli2018twin}, robustness~\citep{pineau2003policy}, explainability~\citep{turner2016model}, traceability~\citep{guo2017semantically, tsirigotis2018orion}, certainty~\citep{diaz2018interactive}, trustworthiness~\citep{xu2017efficient}, transferability~\citep{mehta2019active}, scalability~\citep{luan2019break} and other criteria.

In traditional software engineering, it is reasonable to assume those who are implementing a new system have some implicit domain knowledge and are well-intentioned human beings working towards a common goal -- given a coarse description, they can fill in the blanks. However when building an intelligent system, we would be safer to assume the requirements are implemented by a na\"ive but crafty genie. Given some data and an optimization metric, it will take every available shortcut to satisfy the desired criteria. If we are not careful about correctly specifying the requirements, this entity will create a solution that simply does not work (in the best case), or appears to work but is subtly cursed.

When building an intelligent system developers must carefully ask, ``What are the behavioral requirements of the system?'' This question is often very troublesome, for the requirements cannot be loose specifications, but precise constraints on the solution set. Fully specifying the requirements is indistinguishable from implementing the system -- with the right language abstractions (e.g.\ declarative programming), requirements and implementation can even take the same notation (e.g. SQL, Prolog). But how can we be assured the system satisfies our requirements? Humans can drive a car, but have difficulty describing the algorithm for driving. Labeling the data by hand is too expensive. Exhaustive verification is right out the window. However fuzz testing remains an economical alternative. As we show in \autoref{sec:prob_ad_test}, by making practical assumptions about the model and oracle, we can spend a fixed computational budget to detect more severe errors with lower fiscal and computational overhead.

For example, in the design of a web-based advertisement recommendation system, we can optimize for various criteria such as click rate, engagement, and sales conversion. So long as we can measure these parameters, today's function approximators can optimize for any single criterion or combination thereof (\autoref{eq:moo_spec}). Much of the work involved in machine learning is designing representations which are suitable for downstream tasks and loss functions which accurately measure performance on those tasks. For example, by optimizing for click rate, we create an artificial market for click bots. Similarly, in self-driving vehicles, we often want to optimize for passenger safety. However, by doing so na\"ively can train a vehicle that never moves, or always yields to passing vehicles. Building representations and loss functions which capture the full range of objectives can be a painstaking process using today's toolset.