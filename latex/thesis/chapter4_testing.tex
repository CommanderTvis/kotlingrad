\chapter{Testing intelligent systems}\label{ch:difftest}

\setlength{\epigraphwidth}{0.80\textwidth}
\epigraph{``If we use, to achieve our purposes, a mechanical agency with whose operation we cannot efficiently interfere\ldots then we had better be quite sure the purpose put into the machine is the purpose which we really desire.''}{\begin{flushright}--Norbert \citet{wiener1960some}, \href{https://www.ias.ac.in/article/fulltext/reso/004/01/0080-0088}{\textit{Some moral and technical consequences of automation}}~\end{flushright}}

Today's deep neural networks are capable of learning a broad range of functions, but have specific weaknesses. Training neural networks which can robustly transfer to new domains where the training and test distributions are highly dissimilar poses a significant challenge. These models are often susceptible to failure when presented with carefully crafted inputs. However, the same gradient-based optimization techniques used for training neural networks can also be exploited to probe their failure modes.

In software engineering, techniques for software testing are becoming increasingly automated and general-purpose. Tests help prevent regressive behavior and are a form of specification in which the developer communicates the intended result of running a program. While critically important, tests are often cumbersome to implement. Recent techniques in automated testing have enabled developers to write fewer tests with higher coverage.

In this chapter we propose a novel property-based testing (PBT) algorithm for differentiable programs, and show our method empirical improves sample efficiency over na\"ive probabilistic testing, as measured by its ability to detect a greater proportion of errors violating test constraints in a given budget. Our algorithm can be used to both identify trust region boundaries, and attack a pretrained model given input-output access and a few samples from the training distribution. We further explore the relationship between adversarial methods in machine learning and PBT, and show how adversarial learning can be seen as an extension to a PBT technique known as metamorphic testing (MT).

\section{Background}

In the following sections, we introduce a series of software testing methodologies, in decreasing order of cognitive complexity. We hypothesize the subsequent methods allow developers to attain the same level of assurance with progressively lower effort.

\subsection{Unit Testing}

\noindent In traditional unit testing, each subroutine is accompanied by a single test:
%
\begin{kotlinlisting}
fun unitTest(subroutine: (Input) -> Output) {
    val input = Input() // Construct an input
    val expectedOutput = Output() // Construct an output
    val actualOutput = subroutine(input)
    assert(expectedOutput == actualOutput) { "Expected $expectedOutput, got $actualOutput" }
}
\end{kotlinlisting}
%
Unit testing is an effective to validate one's belief about pre- and post-conditions. The trouble is, someone needs to write a bunch of test cases. Side effects include reduced agility, aversion to refactoring or discarding prior work when tests become obsolete.

\subsection{Integration Testing}

\noindent In integration testing, we are more concerned about the overall behavior of a program, rather than the specific behavior of its subroutines. Consider the following example:

\begin{kotlinlisting}
fun <I, O> integrationTest(program: (I) -> O, inputs: Set<I>, checkOutput: (O) -> Boolean) =
    inputs.forEach { input: I ->
        try {
            val output: O = program(input)
            assert(checkOutput(output)) { "Postcondition failed on $input, $output" }
        } catch (exception: Exception) {
            assert(false) { exception }
        }
    }
\end{kotlinlisting}
%
With this strategy, there are fewer tests to write down, since we only care about end-to-end behavior. Integration testing simply checks a program for terminating exceptions and simple post conditions. For this reason, it is often too coarse-grained.

%For simplicity, in the following sections, we will only consider examples of programs which are pure functions, i.e. which have no external state and produce no side effects.

\subsection{Fuzz Testing}

Fuzz testing is an automated testing methodology which generates random inputs to test a given program. For example, consider the following test:
%
\begin{kotlinlisting}
fun <I, O> fuzzTest(program: (I) -> O, oracle: (I) -> O, rand: () -> I) =
    repeat(1000) {
        val input: I = rand()
        assert(program(input) == oracle(input)) { "Oracle and program disagree on $input" }
    }
\end{kotlinlisting}
%
The trouble is, we need an \textit{oracle}, an often unreasonable assumption. This is known as the \textit{test oracle problem}. But even if we had an oracle, since the space of inputs is often large, it can take a long time to find an output where they disagree. Since a single call to \inline{program(i)} can be quite expensive in practice, this method can also be quite inefficient.

\subsection{Property-based Testing}\label{subsec:property-based-testing}

Property-based testing~\citep{fink1997property} (PBT) attempts to mitigate the test oracle problem by using \textit{properties}. It consists of two phases, searching and shrinking. Users specify a property over all outputs and the test fails if a counterexample can be found:
%
\begin{kotlinlisting}
fun <I, O> gen(program: (I) -> O, property: (O) -> Boolean, rand: () -> I) =
    repeat(1000) {
        val randomInput: I = rand()

        assert(property(program(randomInput))) {
            val shrunken = shrink(randomInput, program, property)
            "Minimal input counterexample of property: $shrunken"
        }
    }
\end{kotlinlisting}
%
Roughly speaking, \inline{shrink} attempts to minimize the counterexample.
%
\begin{kotlinlisting}
tailrec fun <I, O> shrink(failure: I, program: (I) -> O, property: (O) -> Boolean): I =
    if (property(program(decrease(failure)))) failure // Property holds once again
    else shrink(decrease(failure), program, property) // Decrease until property holds
\end{kotlinlisting}
%
For example, given a \inline{program: (Float) -> Any}, we might implement \inline{decrease} like so:
%
\begin{kotlinlisting}
fun decrease(failure: Float): Float = failure - failure / 2
\end{kotlinlisting}
%
\begin{figure}
\begin{tikzpicture}
\begin{axis}[title={Log errors between AD and SD on $f(x) = \frac{\sin(\sin(\sin(x))))}{x} + x\sin(x) + \cos(x) + x$}, width=0.95\textwidth, height=10cm, xlabel=$x$, ylabel=$\log_{10}(\Delta)$, legend pos=south east, align=center]
\addplot table [mark=none, x index=0, y index=1, col sep=comma] {../data/adsd_comparison.csv};
\addlegendentry{$\Delta$(SD, AP) $\approx\Delta$(AD, IP)}
\addplot table [mark=none, x index=0, y index=2, col sep=comma] {../data/adsd_comparison.csv};
\addlegendentry{$\Delta$(AD, SD)}
\addplot table [mark=none, x index=0, y index=3, col sep=comma] {../data/adsd_comparison.csv};
\addlegendentry{$\Delta$(FD, AP)}
\end{axis}
\end{tikzpicture}
\caption{We compare numerical drift between AD and SD over a swollen expression using fixed precision and arbitrary precision (AP). AD and SD both exhibit relative errors (i.e. with respect to each other) several orders of magnitude lower than their absolute error. These results are consistent with the findings of ~\citet{laue2019equivalence}.\vspace{-10pt}}
\label{fig:pbt_comparison}
\end{figure}
%
Consider \autoref{fig:pbt_comparison}, which portrays the log difference between various forms of computational differentiation (evaluated using standard 32-bit precision) and AP (computed to 30 significant figures).\hspace{-.08em}\footnote{To calculate AP, we symbolically derive the function and numerically evaluate it using \hyperref[sec:fdm]{finite difference approximation} and the MacLaurin series expansion of sine and cosine to arbitrary numerical precision.} Given two algorithms for calculating the derivative, a property-based test might check whether the error is bounded over all inputs.

The trouble is, what are the right properties to test? This requiring a lot of effort and domain-specific expertise. In addition, the user must specify a custom shrinker, which is unclear how to implement efficiently. What if there were a better way?

\subsection{Metamorphic testing}\label{subsec:metamorphic-testing}

It is often the case we would like to test the behavior of a program without completely specifying its properties. Many naturally-occurring generative processes exhibit a kind of local invariance -- small changes to the input do not drastically change the output. We can exploit this property to design general-purpose fuzzing methods given just a few inputs and outputs. Metamorphic testing (MT) is an approach to property-based testing which addresses the test oracle problem and the challenge of cheaply discovering bugs in the low-data regime. It has been successfully applied in testing driverless cars~\citep{zhou2019metamorphic, pei2017deepxplore, tian2018deeptest} and other stateful deep learning systems~\citep{du2018deepcruiser}.

First, let us consider the following concrete example, borrowed from \citet{tian2018deeptest}: suppose we have implemented a program which takes an image from a vehicle while driving, and predicts the simultaneous steering angle of the vehicle. Given a single image and the corresponding ground-truth steering angle from an oracle (e.g. a human driver or simulator), our program should preserve invariance under various image transformations, such as limited illumination changes, linear transformations or additive noise below a certain threshold. Intuitively, the steering angle should remain approximately constant, regardless of any single transformation or sequence of transformations applied to the original image which satisfy our chosen criteria. If not, this is a strong indication our program is not sufficiently robust and may not respond well to the sort of variability it may encounter in an operational setting.

Metamorphic testing can be expressed as follows: Given an oracle $\mathbf P: \mathcal I \rightarrow \mathcal O$, and a set of inputs $\mathbf X = \{\mathbf{x}^{(1)}, \dots, \mathbf{x}^{(z)}\}$ and outputs $\mathbf Y = \{\mathbf{y}^{(1)} = \mathbf{P}(\mathbf{x}^{(1)}), \dots, \mathbf{y}^{(z)} = \mathbf{P}(\mathbf{x}^{(z)})\}$, a metamorphic relation (MR) is a relation $\mathcal R \subset \mathcal I^z \times \mathcal O^z$ where $z \geq 2$. In the simplest case, an MR is an equivalence relation $\mathcal R$, i.e.: $\langle \mathbf x, \mathbf y, \mathbf x', \mathbf y' \rangle \in \mathcal R \Leftrightarrow \mathbf x \sim_{\mathcal R} \mathbf x' \Leftrightarrow \mathbf P(\mathbf x) \approx \mathbf P(\mathbf x')$.

Suppose our MR is $\forall \varphi \in \mathcal I: ||\mathbf\varphi|| \leq \varepsilon, \mathbf P(\mathbf x) \approx \mathbf P(\mathbf x' = \mathbf x + \varphi) \approx \mathbf y$. Given a program $\mathbf{\hat P}$ and a comparatively small set of inputs $\mathbf X$ and outputs $\mathbf Y$ from our oracle $\mathbf P$, the MR produces a set $\mathbf X', |\mathbf X| \ll |\mathbf X'|$ on which to test $\mathbf{\hat P}$, without requiring corresponding outputs from $\mathbf P$. If we can show $\exists \mathbf x' \in \mathbf X' \mid \mathbf{\hat P}(\mathbf x') \not\approx \mathbf P(\mathbf x)$, this implies at least one of the following:

\begin{enumerate}
\item $\langle \mathbf x, \mathbf P(\mathbf x), \mathbf x', \mathbf P(\mathbf x')\rangle \notin \mathcal R$, i.e. our assumptions were invalid
\item $\mathbf{\hat P}(\mathbf x') \not\approx \mathbf{P}(\mathbf x')$, i.e. the program under test is unsound
\end{enumerate}
%
In either case, we have obtained useful information. If our assumptions were invalid, we can strengthen the invariant, $\mathcal R$, by removing the counterexample. Otherwise, we have detected an error and can adjust the program to ensure compliance -- both are useful outcomes.

Consider the following example of an MT which uses an equivalence-based MR:

\begin{kotlinlisting}
fun <I, O> mrTest(program: (I) -> O, mr: (I, O, I, O) -> Boolean, rand: () -> Pair<I, O>) =
    repeat(1000) {
        val (input: I, output: O) = rand()
        val tx: (I) -> I = genTX(program, mr, input, output)
        val txInput: I = tx(input)
        val txOutput: O = program(txInput)
        assert(mr(input, output, txInput, txOutput)) {
            "<$input, $output> not related to <$txInput, $txOutput> by $mr ($tx)"
        }
    }
\end{kotlinlisting}
%
The trouble is, generating valid transformations is a non-trivial exercise. We could try to generate random transformations until we find one which meets our criteria:
%
\begin{kotlinlisting}
fun <I, O> genTX(program: (I) -> O, mr: (I, O, I, O) -> Boolean, i: I, o: O): (I) -> I {
    while (true) {
        val tx: (I) -> I = sampleRandomTX()
        val txInput: I = tx(i)
        val txOutput: O = program(txInput)
        if (mr(i, o, txInput, txOutput)) return tx
    }
}
\end{kotlinlisting}

But this would be very inefficient and depending on the type of input and output, is not guaranteed to terminate. We could handcraft a transformation, but this requires extensive domain knowledge. Instead, we should seek a more principled, computationally efficient and general purpose method of mutating an input in our dataset to discover invalid outputs.

\subsection{Adversarial Testing}

This leads us to adversarial testing. In the general case, we are given an input-output pair from an oracle and a program approximating the oracle, but not necessarily the oracle itself. Our goal is to find a small change to the input of a function, which produces the largest change to its output, relative to the original output.

Imagine a function $\mathbf{\hat P}: \mathbb R^m \rightarrow \mathbb R$, each component $g_1, ..., g_{m}$ of which we seek to change by a fixed amount so as to produce the largest output value $\mathbf{\hat P}(g'_1, ..., g'_{m})$ directly. Suppose for each input parameter $g_1, \ldots, g_{m}$, we have one of three choices to make: either we can increase the value by $c$, decrease the value by $c$, or leave it unchanged. We are given no further information about $\mathbf{\hat P}$. Consider the na\"ive solution, which tries every combination of variable perturbations and selects the input corresponding to the greatest output value:

\begin{algorithm}[H]
\caption{Brute Force Adversary}
\label{alg:bf_adversary}
\begin{algorithmic}[1]
\Procedure{BfAdversary}{$\mathbf{\hat P}: \mathbb{R}^m \rightarrow \mathbb{R}$, $c: \mathbb R$, $g_1: \mathbb R$, $g_2: \mathbb R$, $\ldots$, $g_{m}: \mathbb R$}: $\mathbb{R}^m$
\If {$m = 1$} \Comment{Evaluate $\mathbf{\hat P}$ and return the best variable perturbation}
\State \Return $\operatorname{argmax}\{\mathbf{\hat P}(g_1 + c), \mathbf{\hat P}(g_1 - c), \mathbf{\hat P}(g_1)\}$
\Else \Comment{Partially apply candidate perturbation and recurse}
\State \Return $\operatorname{argmax}\{\mathbf{\hat P}(g_1 + c) \circ$\Call{BfAdversary}{$\mathbf{\hat P}(g_1 + c), c, g_2, \ldots, g_{m}$},\newline
\hspace*{10em} $\mathbf{\hat P}(g_1 - c)\circ$\Call{BfAdversary}{$\mathbf{\hat P}(g_1 - c), c, g_2, \ldots, g_{m}$},\newline
\hspace*{10em} $\mathbf{\hat P}(g_1)\circ$\Call{BfAdversary}{$\mathbf{\hat P}(g_1), c, g_2, \ldots, g_{m}$}$\}$
\EndIf
\EndProcedure
\end{algorithmic}
\end{algorithm}

As we can see, \autoref{alg:bf_adversary} is $\mathcal{O}(3^m)$ with respect to $\dim \mathbf g$ -- not a very efficient search routine, especially if we want to consider a larger set of perturbances. Clearly, if we want to find the best direction to update $\mathbf g$, we need to be more careful when performing the search.

Even if we cannot compute a closed-form for $\nabla_{\mathbf g}\mathbf{\hat P}$, if $\mathbf{\hat P}$ is differentiable almost everywhere, we can still use numerical differentiation to approximate pointwise values of the gradient. Consider \autoref{alg:fd_fuzz}, which uses the \hyperref[sec:fdm]{finite difference method} to approximate $\nabla_{\mathbf g}\mathbf{\hat P}$. This tells us how to minimally change the input to produce the largest output in reach, without needing to exhaustively check every perturbation as in \autoref{alg:bf_adversary}.

\begin{algorithm}[H]
\caption{Finite Difference Adversary}
\label{alg:fd_fuzz}
\begin{algorithmic}[1]
\Procedure{FdAdversary}{$\mathbf{\hat P}: \mathbb{R}^m \rightarrow \mathbb{R}$, $c: \mathbb R$, $g_1: \mathbb R$, $g_2: \mathbb R$, $\ldots$, $g_{m}: \mathbb R$}: $\mathbb{R}^m$
\If {$m = 1$} \Comment{Compute finite (centered) difference and perform gradient ascent}
\State \Return $g_1 + \frac{\mathbf{\hat P}(g_1 - c) - \mathbf{\hat P}(g_1 + c)}{2c}$
\Else \Comment{Apply single-step gradient ascent using componentwise finite difference}
\State \Return $g_1 + \frac{\mathbf{\hat P}(g_1 - c, 0, \ldots) - \mathbf{\hat P}(g_1 + c, 0, \ldots)}{2c}$, \Call{FdAdversary}{$\mathbf{\hat P}, c, g_2, \ldots, g_{m}$}
\EndIf
\EndProcedure
\end{algorithmic}
\end{algorithm}

We now have a procedure that is $\mathcal{O}(m)$ with respect to $\mathbf{\hat P}$, but must be recomputed for each input -- we can still do better by assuming further structure on $\mathbf{\hat P}$. Furthermore, we have not yet incorporated any constraint on the input values. Perhaps we can combine the notion of metamorphic testing seen in \autoref{subsec:metamorphic-testing} with constrained optimization to accelerate the search for adversarial examples.

During backpropagation we perform gradient descent on a differentiable function with respect to its parameters for a specific set of inputs. In gradient-based adversarial testing, we perform gradient ascent on a loss function with respect to the inputs using a fixed parameter setting. Suppose we have a differentiable vector function $\mathbf{\hat P}: \mathbb{R}^m\rightarrow\mathbb{R}^n$, defined as follows:
%
\begin{equation} \tag{\autoref{eq:recursive_parametric_eq} revisited}
\mathbf{\hat P}_k(\mathbf{x}; \bm\Theta) = \begin{cases} \mathbf{\hat p}_1(\Theta_1)\circ\mathbf{x} &\text{if } k=1\\ \mathbf{\hat p}_k(\Theta_k)\circ \mathbf{\hat P}_{k}(\bm\Theta_{[1, k]})\circ\mathbf{x}&\text{if } k > 1 \end{cases} \\
\end{equation}
%
In deep learning, given pairs $\mathbf{X} = \{\mathbf{x}^{(1)}, \dots, \mathbf{x}^{(z)}\}, \mathbf{Y} = \{\mathbf{y}^{(1)} = \mathbf{P}(\mathbf{x}^{(1)}), \dots, \mathbf{y}^{(z)} = \mathbf{P}(\mathbf{x}^{(z)})\}$ we want to find $\bm\Theta^* = \argmin{\boldsymbol{\Theta}}\mathcal{L}\big(\mathbf{\hat P}_k(\mathbf{x}^{(i)}; \bm\Theta), \mathbf{y}^{(i)}\big)$ which is typically achieved by performing stochastic gradient descent on the loss with respect to the model parameters:
%
\begin{equation} \tag{\autoref{eq:stochastic_grad_descent} revisited}
\bm\Theta \leftarrow \bm\Theta - \alpha\frac{1}{z}\nabla_{\bm\Theta} \sum_{i=1}^z\mathcal{L}\big(\mathbf{\hat P}_k(\mathbf{x}^{(i)}; \bm\Theta), \mathbf{y}^{(i)}\big)
\end{equation}
%
We can solve for the gradient with respect to $\bm\Theta$ by multiplying the Jacobians (\autoref{eq:vfun_chain_rule}), $\mathcal{J}_{\mathbf{p}_1} \cdots \mathcal{J}_{\mathbf{p}_k}$. In white box adversarial learning, we are given a fixed $\bm\Theta$~\footnote{In contrast with backpropagation, where the parameters $\bm\Theta$ are updated.} and control the value of $\mathbf x$, so we can rewrite $\mathbf{\hat P}_k(\mathbf{x}^{(i)};\bm\Theta)$ instead as $\mathbf{\hat P}(\mathbf x)$, and take the gradient directly with respect to $\mathbf x$. Our objective is to find the ``worst'' $\mathbf x$ within a small distance of any $\mathbf x^{(i)}$, i.e. where $\mathbf{P}(\mathbf x)$ least resembles $\mathbf{\hat P}(\mathbf x)$. More concretely, this can be expressed as,
%
\begin{equation}
\mathbf{x}^* = \argmax{\mathbf{x}}\mathcal{L}\big(\mathbf{\hat P}(\mathbf{x}), \mathbf{y}^{(i)}\big) \text{ subject to } CS = \{\mathbf{x} \in \mathbb{R}^m \text{ s.t. } ||\mathbf{x}^{(i)} - \mathbf{x}||_p    < \epsilon\}
\end{equation}
%
To do so, we can initialize $\mathbf{x} \sim U[CS]$ and perform projected gradient ascent on the loss:
%
\begin{equation}\label{eq:projected_gd}
\mathbf x \leftarrow \bm\Phi_{CS}\Big(\mathbf x + \alpha\mathbf\nabla_{\mathbf x} \mathcal{L}\big(\mathbf{\hat P}(\mathbf{x}), \mathbf{y}^{(i)}\big)\Big) \text{, where }
\bm\Phi_{CS}(\mathbf \phi') = \argmin{\mathbf \phi \in CS}\frac{1}{2}||\mathbf \phi - \mathbf \phi'||^2_2
\end{equation}
%

Assuming zero knowledge about the program $\mathbf{\hat P}$'s implementation or data distribution, $D_{\mathbf{\hat P}}$, we can do no better than random search~\citep{wolpert1997no}. Assuming $\mathbf{\hat P}$ is differentiable, given input-output values we can use zeroth-order optimization techniques to approximate $\nabla_{\mathbf{x}}\mathcal{L}$. Assuming $\mathbf{\hat P}$ is open source, we could use coverage-guided fuzzing to prioritize the search for inputs more likely to violate $\mathbf T$. If $\mathbf{\hat P}$ is both open source and differentiable, we can accelerate the search by using automatic differentiation. Given additional information about the training distribution, we could initialize the search in unseen regions of the input space, e.g. sample from the inverse distribution $\mathbf x \sim \frac{1}{D_{\mathbf{\hat P}}}$, possibly more likely to elicit an error. But all this requires a great deal of human expertise to implement efficiently. What if it were possible to generate an adversary instead of manually constructing one?

\subsection{Generative Adversarial Testing}

What are the properties of a good adversary? For an adversary to be considered a strong adversary, a significant fraction of her attacks must break the program specification. To generate plausible test cases, not only must she be able to exploit weaknesses of the program, but ideally possess a good understanding of $p_{data}$.

Suppose we have a program $D: \mathbb{R}^h\rightarrow\mathbb{B}$, i.e. a binary classifier. How should we attack its implementation, without a custom adversary, or defining some prior distribution over the inputs? One solution, known as a generative adversarial network~\citep{goodfellow2014gan} (GAN), proposes to train a ``generative'' adversary $G: \mathbb{R}^v\rightarrow\mathbb{R}^h$ alongside the trained model. The vanilla GAN objective can be expressed as a minimax optimization problem:

\begin{equation}
\min_G \max_D V(D, G) = \mathbb{E}_{\mathbf x \sim p_{data}}\big[\log D(\mathbf x)\big] + \mathbb{E}_{\mathbf z \sim p_{\mathbf z}}\Big[\log\Big(1 - D\big(G(\mathbf z^{(i)})\big)\Big)\Big]
\end{equation}

This objective can be sought by sampling minibatches $\mathbf x \sim p_{data}$ and $\mathbf z \sim p_{G}$, then updating the parameters of $G$ and $D$ using their respective stochastic gradients:

\begin{equation}
\bm\Theta_D \leftarrow \bm\Theta_D + \nabla_{\bm\Theta_D}\frac{1}{m}\sum_{i=1}^m\Big[\log D(\mathbf x^{(i)}) + \log\Big(1 - D\big(G(\mathbf z^{(i)})\big)\Big)\Big]
\end{equation}

\begin{equation}
\bm\Theta_G \leftarrow \bm\Theta_G - \nabla_{\bm\Theta_G}\frac{1}{m}\sum_{i=1}^m \log\Big(1 - D\big(G(\mathbf z^{(i)})\big)\Big)
\end{equation}
%
\citet{albuquerque2019hgan} propose an augmented version of this game using multiple Discriminators which each receive a fixed, random projection $P_k(\cdot)$ of the Generator's output, and solves the following multi-objective optimization problem:
%
\begin{equation} \label{eq:moo_spec}
\min \mathbf{\mathcal{L}}_G(\mathbf x) = \big[l_1(\mathbf z), l_2(\mathbf z), \ldots, l_K(\mathbf z)\big] \text{, where } l_k = -\mathbb E_{z \sim p_z} \log D_k\Big(P_k\big(G(z_k)\big)\Big)
\end{equation}
%
This can be solved by combining the losses using a form of hypervolume maximization:
%
\begin{equation}
\nabla_{\bm\Theta} \mathcal{L}_G = \sum_{k=1}^K \frac{1}{\eta - l_k}\nabla_\Theta l_k
\end{equation}

Where $\eta$ is a common, fixed upper bound on every $l_k$. Further GAN variants such as WGAN~\citep{arjovsky2017wgan}, MHGAN~\citep{turner2019mhgan}, et al. have proposed augmentations to the vanilla GAN to improve stability and sample diversity. GANs have been successfully applied in various domains from speech~\citep{donahue2019wavegan} to graph synthesis~\citep{wang2018graphgan}. One practical extension to the latter could be applying the GAN framework to program synthesis and compiler optimization by choosing a suitable metric and following the approach proposed by e.g.~\citet{adams2019learning, mendis2019compiler}.

The trouble with GANs is that we need to train $G$ and $D$ in lockstep, otherwise one quickly becomes too strong. What happens if we want to attack a pretrained model?
%
\section{Probabilistic Adversarial Testing}\label{sec:prob_ad_test}

\noindent Henceforth we shall refer to $\mathcal{L}\big(\mathbf{\hat P}(\mathbf{x}), \mathbf{y}\big)$ as $\mathcal{L}(\mathbf x)$. Imagine a single test $\mathbf{T}: \mathbb{R}^m \rightarrow \mathbb{B}$:
%
\begin{equation} \label{eq:output_constraint_example}
\mathbf T(\mathbf{x}) = \mathcal{L}(\mathbf{x}) < C
\end{equation}
%
Our goal is to find a set of inputs which break our test given a computational budget $\mathcal{B}_e$ (i.e.\ fixed number of program evaluations) and labeling budget $\mathcal{B}_l$ (i.e.\ fixed number of labels).
%
\begin{equation}
\{ D_\mathbf T: \mathbf x \in CS \mid \mathcal{L}(\mathbf{x}) < C\}, \text{ maximize } |D_\mathbf T| \text { subject to } \mathcal{B}_e, \mathcal{B}_l
\end{equation}
%

Let us consider an extension of classical fuzzing methods to differentiable functions on continuous random variables. First, we sample an input $\mathbf{x}_j: \mathbb{R}^m \sim \mathcal S_m$ (e.g.\ uniformly). If $\mathcal{L}(\mathbf{x}_j)$ satisfies \autoref{eq:output_constraint_example}, we ascend the loss following $\nabla_{\mathbf x}\mathcal{L}$, otherwise we descend and repeat until the test ``flips'', gradient vanishes, or a fixed number of steps $I_{max}$ are reached before resampling $\mathbf{x}_{j+1}$ from $\mathcal S_m$. This procedure is described in \autoref{alg:prob_adversary}.

We hypothesize that if $\mathbf{\hat P}$'s implementation were flawed and a counterexample to \autoref{eq:output_constraint_example} existed, as sample size increased, a subset of trajectories would fail to converge at all, a subset would converge to local optima, and the remaining trajectories would discover the boundary.

\begin{algorithm}[ht]
\caption{Probabilistic Generator}
\label{alg:prob_adversary}
\begin{algorithmic}[1]
\Procedure{ProbGen}{$\mathcal L: \mathbb R^m \rightarrow \mathbb R$, $\mathcal S_m$, $\mathbf{T}: \mathbb{R}^m \rightarrow \mathbb{B}$, $\mathcal{B}_e: \mathbb R$}
\State $D_\mathbf T \gets \{\}, j \gets 0$
\While{$0 < \mathcal{B}_e$}\Comment{Iterate until computational budget exhausted}
\State $\mathbf{x}_j \sim \mathcal S_m$\Comment{Sample from $\mathcal S_m$}
\If {$\mathbf T\big(\mathbf{x}_j, C\big)$} \Comment{Inside feasible set, perform gradient ascent}
\State $D_\mathbf T \gets D_\mathbf T$ $\cup$ \Call{DiffShrink}{$-\mathcal L, \mathbf{x}_j, \mathbf T$}
\Else \Comment{Outside feasible set, perform gradient descent}
\State $D_\mathbf T \gets D_\mathbf T$ $\cup$  \Call{DiffShrink}{$\mathcal L, \mathbf{x}_j, \mathbf T$}
\EndIf
\State $\mathcal{B}_e \gets \mathcal{B}_e - 1$
\EndWhile
\State \Return $D_\mathbf T$
\EndProcedure
\end{algorithmic}
\end{algorithm}

\begin{algorithm}[ht]
\caption{Differential Shrinker}
\label{alg:diff_adversary}
\begin{algorithmic}[1]
\Procedure{DiffShrink}{$\mathcal L: \mathbb R^m \rightarrow \mathbb R$, $\mathbf{x}_1: \mathbb R^m$, $\mathbf{T}: \mathbb{R}^m \rightarrow \mathbb{B}$}
\State $i \gets 1, t_i \gets \mathbf T\big(\mathbf x_i, C\big)$ \Comment{Store initial state to detect when test flips.}
\Do
    \State $i \gets i + 1, \mathbf x_i \gets \bm\Phi_{CS}\big(\mathbf x_{i-1} - \alpha\mathbf\nabla_{\mathbf x} \mathcal{L}(\mathbf{x}_{i-1})\big)$ \Comment{PGD step (\autoref{eq:projected_gd})}
    \If {$\mathbf T\big(\mathbf{x}_i, C\big) \neq t_1$} \Comment{Boundary value was found.}
    \State \Return \textbf{if } $t_1$ \textbf{ then } $\{\mathbf{x}_{i}\}$ \textbf{ else } $\{\mathbf{x}_{i-1}\}$ \textbf{ end if} \Comment{Always return violation.}
    \EndIf
\doWhile {$i \leq I_{max}$ \textbf{and} $\epsilon < \mathbf |\mathcal{L}(\mathbf x_i) - \mathcal{L}(\mathbf x_{i-1})|$} \Comment{While not converged.}
\State \Return \textbf{if } $\neg t_1$ \textbf{ then } $\{\mathbf x_{i-1}\}$ \textbf{ else } $\varnothing$ \Comment{Return last iterate or $\varnothing$ if test passed.}
\EndProcedure
\end{algorithmic}
\end{algorithm}

We evaluate our algorithm in the regression setting, where $\mathbf{\hat P}$ is a polynomial regressor (cf. \autoref{sec:poly_reg}) and $\mathcal{L}$ is the mean squared error loss.

 Our training set consists of input-output pairs from a set of random algebraic expressions. These expressions are produced by generating perfect binary trees of depth 5, whose leaf nodes contain with equal probability either (1) an alphabetic variable or (2) a random 64-bit IEEE 754 floating point number uniformly sampled in the range $[-1, 1]$. The internal nodes contain with equal probability a random operator in the set $\{+, \times\}$. Our expression generator (\autoref{eq:btree_gen}) with type $G_e: \mathbb{N}^+\times\mathbb{Z} \rightarrow \mathbb{R}^{[1, 26]} \rightarrow \mathbb{R}$ takes a depth $\delta: \mathbb{N}^+$, a random seed $\psi: \mathbb{Z}$, and returns a scalar-valued function.

\begin{equation}\label{eq:btree_gen}
G_e(\delta, \psi) = \begin{cases}
    \delta \leq 0 \begin{cases}
    \delta\sim_\psi\{a,b,..z\} \text{ if } \gamma\sim_\psi\{\text{True, False}\},\\
    \chi\sim_\psi U(-1, 1) \text{ otherwise.}
    \end{cases} \\
    \delta > 0 \begin{cases}
    G(\delta-1, \psi + 1) + G(\delta-1, \psi - 1) \text{ if } \gamma\sim_\psi\{\text{True, False}\},\\
    G(\delta-1, \psi + 1) \times G(\delta-1, \psi - 1) \text{ otherwise.}
    \end{cases}
\end{cases}
\end{equation}

A Kotlin implementation of the expression tree generator in \autoref{eq:btree_gen} is shown below:
%
\begin{kotlinlisting}
val sum = { left: SFun<DReal>, right: SFun<DReal> -> left + right }
val mul = { left: SFun<DReal>, right: SFun<DReal> -> left * right }
val operators = listOf(sum, mul)
val variables = ('a'..'z').map { SVar<DReal>(it) }

infix fun SFun<DReal>.wildOp(that: SFun<DReal>) = operators.random(rand)(this, that)

fun randomBiTree(height: Int): SFun<DReal> =
  if (height == 0) (listOf(wrap(rand.nextDouble(-1.0, 1.0))) + variables).random(rand)
  else randomBiTree(height - 1) wildOp randomBiTree(height - 1)
\end{kotlinlisting}

Our training set consists of input-output pairs produced by binding the set of free variables to values, and numerically evaluating the expression on input values sampled from $[-1, -0.2] \cup [0.2, 1]$, then rescaling all outputs to $[-1, 1]$ using min-max normalization, i.e. $\tilde{G}_e(\delta, \psi)= \frac{G_e(\delta, \psi)}{\max |G_e(\delta, \psi)[-1, 1]|}$. Each expression has a unique validation set $x_i \sim [-0.2, 0.2]$.

In \autoref{fig:loss_curves}, we see train and validation losses for 200 trajectories of momentum SGD through parameter space. To compensate for the difference in magnitude between training and validation error, we normalize all losses by their respective values at $t_0$. Based on the validation loss, we apply early stopping at approximately 50 epochs.

\begin{algorithm}[ht]
\caption{Surrogate Attack}
\label{alg:surrogate_attack}
\begin{algorithmic}[1]
\Procedure{SurrogateAttack}{$\bm\Theta: \mathbb{R}^k$, $\mathbf{\hat f}: \mathbb R^m \times \mathbb R^k\rightarrow \mathbb R^n$, $\mathcal S_m$, $\mathbf{T}: \mathbb{R}^m \rightarrow \mathbb{B}$, $\mathcal{B}_l: \mathbb R$}
\State $\bm\Theta' \gets \bm\Theta$
\Do
    \State $\mathbf{x} \sim \mathcal S_m, \mathbf{y} \gets \mathcal{O}(\mathbf{x}), \mathcal{B}_l \gets \mathcal{B}_l - 1$ \Comment{Ask for new label from the oracle.}
    \State $\bm\Theta' \gets \bm\Theta - \alpha\nabla_{\bm\Theta}||\mathbf{\hat f}(\mathbf{x}; \bm\Theta') - \mathbf{y}||_2,$ \Comment{Update parameters using loss gradient.}
\doWhile {$0 < \mathcal{B}_l$} \Comment{Iterate until labeling budget exhausted.}
\State $\mathcal{\hat{L}} \gets ||\mathbf{\hat f}(\bm\Theta') - \mathbf{\hat f}(\bm\Theta)||_2$ \Comment{Construct the surrogate loss.}
\State \Return \Call{ProbGen}{$\mathcal{\hat{L}}, \mathcal S_m, \mathbf T, \mathcal{B}_e$}
\EndProcedure
\end{algorithmic}
\end{algorithm}

\begin{figure}[H]
\centering
\begin{tikzpicture}
\begin{axis}[title={Average loss curves for polynomial regression using momentum SGD}, width=\textwidth, height=10cm, xlabel=Epochs, ylabel=$t_0$-normalized MSE, legend pos=north east, align=center]
\addplot[smooth, blue] table [mark=none, x=x, y=y1, col sep=comma] {../data/poly_train_loss.csv};
\addlegendentry{Training}
\addplot[smooth, red] table [mark=none, x=x, y=y2, col sep=comma] {../data/poly_train_loss.csv};
\addlegendentry{Validation}
\addplot [smooth, name path=upper1,draw=none] table[x=x, y1=y1, y1_err=y1_err, y expr=\thisrow{y1}+\thisrow{y1_err}, col sep=comma] {../data/poly_train_loss.csv};
\addplot [smooth, name path=lower1,draw=none] table[x=x, y1=y1, y1_err=y1_err, y expr=\thisrow{y1}-\thisrow{y1_err}, col sep=comma] {../data/poly_train_loss.csv};
\addplot [fill=blue!10] fill between[of=upper1 and lower1];
\addplot [smooth, name path=upper2,draw=none] table[x=x, y2=y2, y2_err=y2_err, y expr=\thisrow{y2}+\thisrow{y2_err}, col sep=comma] {../data/poly_train_loss.csv};
\addplot [smooth, name path=lower2,draw=none] table[x=x, y2=y2, y2_err=y2_err, y expr=\thisrow{y2}-\thisrow{y2_err}, col sep=comma] {../data/poly_train_loss.csv};
\addplot [fill=red!10] fill between[of=upper2 and lower2];
\end{axis}
\end{tikzpicture}
\caption{For each expression in our dataset, we train a polynomial regressor to convergence.}
\label{fig:loss_curves}
\end{figure}

\begin{table}[H]
\begin{tabular}{ll}
\digraph[scale=0.2]{btree0} {
    graph ["rankdir"="LR","bgcolor"="transparent"];
    "97652294" ["color"="black","fontcolor"="black","fontname"="Helvetica","fontsize"="20","style"="setlinewidth(2)","label"="*"];
    "1027007693" ["color"="black","fontcolor"="black","fontname"="Helvetica","fontsize"="20","style"="setlinewidth(2)","label"="*"];
    "282432134" ["color"="black","fontcolor"="black","fontname"="Helvetica","fontsize"="20","style"="setlinewidth(2)","label"="+"];
    "1282811396" ["color"="black","fontcolor"="black","fontname"="Helvetica","fontsize"="20","style"="setlinewidth(2)","label"="+"];
    "641853239" ["color"="black","fontcolor"="black","fontname"="Helvetica","fontsize"="20","style"="setlinewidth(2)","label"="+"];
    "1920467934" ["color"="black","fontcolor"="black","fontname"="Helvetica","fontsize"="20","style"="setlinewidth(2)","label"="*"];
    "1883840933" ["color"="black","fontcolor"="black","fontname"="Helvetica","fontsize"="20","style"="setlinewidth(2)","label"="*"];
    "233996206" ["color"="black","fontcolor"="black","fontname"="Helvetica","fontsize"="20","style"="setlinewidth(2)","label"="0.307"];
    "x" ["color"="black","fontcolor"="black","fontname"="Helvetica","fontsize"="20","style"="setlinewidth(2)"];
    "614685048" ["color"="black","fontcolor"="black","fontname"="Helvetica","fontsize"="20","style"="setlinewidth(2)","label"="+"];
    "385337537" ["color"="black","fontcolor"="black","fontname"="Helvetica","fontsize"="20","style"="setlinewidth(2)","label"="-"];
    "265119009" ["color"="black","fontcolor"="black","fontname"="Helvetica","fontsize"="20","style"="setlinewidth(2)","label"="*"];
    "832279283" ["color"="black","fontcolor"="black","fontname"="Helvetica","fontsize"="20","style"="setlinewidth(2)","label"="+"];
    "789219251" ["color"="black","fontcolor"="black","fontname"="Helvetica","fontsize"="20","style"="setlinewidth(2)","label"="-"];
    "1375995437" ["color"="black","fontcolor"="black","fontname"="Helvetica","fontsize"="20","style"="setlinewidth(2)","label"="+"];
    "1434041222" ["color"="black","fontcolor"="black","fontname"="Helvetica","fontsize"="20","style"="setlinewidth(2)","label"="+"];
    "838411509" ["color"="black","fontcolor"="black","fontname"="Helvetica","fontsize"="20","style"="setlinewidth(2)","label"="*"];
    "1338841523" ["color"="black","fontcolor"="black","fontname"="Helvetica","fontsize"="20","style"="setlinewidth(2)","label"="-"];
    "802581203" ["color"="black","fontcolor"="black","fontname"="Helvetica","fontsize"="20","style"="setlinewidth(2)","label"="*"];
    "929776179" ["color"="black","fontcolor"="black","fontname"="Helvetica","fontsize"="20","style"="setlinewidth(2)","label"="-"];
    "2050404090" ["color"="black","fontcolor"="black","fontname"="Helvetica","fontsize"="20","style"="setlinewidth(2)","label"="+"];
    "1561408618" ["color"="black","fontcolor"="black","fontname"="Helvetica","fontsize"="20","style"="setlinewidth(2)","label"="+"];
    "668210649" ["color"="black","fontcolor"="black","fontname"="Helvetica","fontsize"="20","style"="setlinewidth(2)","label"="0.355"];
    "1545087375" ["color"="black","fontcolor"="black","fontname"="Helvetica","fontsize"="20","style"="setlinewidth(2)","label"="0.188"];
    "388043093" ["color"="black","fontcolor"="black","fontname"="Helvetica","fontsize"="20","style"="setlinewidth(2)","label"="0.886"];
    "188576144" ["color"="black","fontcolor"="black","fontname"="Helvetica","fontsize"="20","style"="setlinewidth(2)","label"="-1.08"];
    "266437232" ["color"="black","fontcolor"="black","fontname"="Helvetica","fontsize"="20","style"="setlinewidth(2)","label"="0.188"];
    "500772834" ["color"="black","fontcolor"="black","fontname"="Helvetica","fontsize"="20","style"="setlinewidth(2)","label"="1.0"];
    "1691538257" ["color"="black","fontcolor"="black","fontname"="Helvetica","fontsize"="20","style"="setlinewidth(2)","label"="pow"];
    "1335505684" ["color"="black","fontcolor"="black","fontname"="Helvetica","fontsize"="20","style"="setlinewidth(2)","label"="1.555"];
    "1226204845" ["color"="black","fontcolor"="black","fontname"="Helvetica","fontsize"="20","style"="setlinewidth(2)","label"="-"];
    "1" ["color"="black","fontcolor"="black","fontname"="Helvetica","fontsize"="20","style"="setlinewidth(2)","label"="One"];
    "1027007693" -> "97652294" ["color"="blue","arrowhead"="normal","style"="setlinewidth(2)"];
    "282432134" -> "1027007693" ["color"="blue","arrowhead"="normal","style"="setlinewidth(2)"];
    "1282811396" -> "282432134" ["color"="blue","arrowhead"="normal","style"="setlinewidth(2)"];
    "641853239" -> "1282811396" ["color"="blue","arrowhead"="normal","style"="setlinewidth(2)"];
    "1920467934" -> "641853239" ["color"="blue","arrowhead"="normal","style"="setlinewidth(2)"];
    "1883840933" -> "1920467934" ["color"="blue","arrowhead"="normal","style"="setlinewidth(2)"];
    "233996206" -> "1883840933" ["color"="blue","arrowhead"="normal","style"="setlinewidth(2)"];
    "x" -> "1883840933" ["color"="red","arrowhead"="normal","style"="setlinewidth(2)"];
    "x" -> "614685048" ["color"="blue","arrowhead"="normal","style"="setlinewidth(2)"];
    "x" -> "385337537" ["color"="black","arrowhead"="normal","style"="setlinewidth(2)"];
    "x" -> "265119009" ["color"="red","arrowhead"="normal","style"="setlinewidth(2)"];
    "x" -> "1375995437" ["color"="blue","arrowhead"="normal","style"="setlinewidth(2)"];
    "x" -> "1338841523" ["color"="black","arrowhead"="normal","style"="setlinewidth(2)"];
    "x" -> "802581203" ["color"="blue","arrowhead"="normal","style"="setlinewidth(2)"];
    "x" -> "802581203" ["color"="red","arrowhead"="normal","style"="setlinewidth(2)"];
    "x" -> "2050404090" ["color"="blue","arrowhead"="normal","style"="setlinewidth(2)"];
    "614685048" -> "1920467934" ["color"="red","arrowhead"="normal","style"="setlinewidth(2)"];
    "385337537" -> "614685048" ["color"="red","arrowhead"="normal","style"="setlinewidth(2)"];
    "265119009" -> "832279283" ["color"="blue","arrowhead"="normal","style"="setlinewidth(2)"];
    "832279283" -> "789219251" ["color"="black","arrowhead"="normal","style"="setlinewidth(2)"];
    "789219251" -> "641853239" ["color"="red","arrowhead"="normal","style"="setlinewidth(2)"];
    "1375995437" -> "1434041222" ["color"="blue","arrowhead"="normal","style"="setlinewidth(2)"];
    "1434041222" -> "838411509" ["color"="blue","arrowhead"="normal","style"="setlinewidth(2)"];
    "838411509" -> "1282811396" ["color"="red","arrowhead"="normal","style"="setlinewidth(2)"];
    "1338841523" -> "1375995437" ["color"="red","arrowhead"="normal","style"="setlinewidth(2)"];
    "802581203" -> "929776179" ["color"="black","arrowhead"="normal","style"="setlinewidth(2)"];
    "929776179" -> "1434041222" ["color"="red","arrowhead"="normal","style"="setlinewidth(2)"];
    "2050404090" -> "1561408618" ["color"="blue","arrowhead"="normal","style"="setlinewidth(2)"];
    "1561408618" -> "838411509" ["color"="red","arrowhead"="normal","style"="setlinewidth(2)"];
    "668210649" -> "265119009" ["color"="blue","arrowhead"="normal","style"="setlinewidth(2)"];
    "1545087375" -> "832279283" ["color"="red","arrowhead"="normal","style"="setlinewidth(2)"];
    "388043093" -> "2050404090" ["color"="red","arrowhead"="normal","style"="setlinewidth(2)"];
    "188576144" -> "1561408618" ["color"="red","arrowhead"="normal","style"="setlinewidth(2)"];
    "266437232" -> "282432134" ["color"="red","arrowhead"="normal","style"="setlinewidth(2)"];
    "500772834" -> "1027007693" ["color"="red","arrowhead"="normal","style"="setlinewidth(2)"];
    "1691538257" -> "97652294" ["color"="red","arrowhead"="normal","style"="setlinewidth(2)"];
    "1335505684" -> "1691538257" ["color"="blue","arrowhead"="normal","style"="setlinewidth(2)"];
    "1226204845" -> "1691538257" ["color"="red","arrowhead"="normal","style"="setlinewidth(2)"];
    "1" -> "1226204845" ["color"="black","arrowhead"="normal","style"="setlinewidth(2)"];
} & \begin{tikzpicture} \begin{axis}[title={}, width=0.4\textwidth, height=5cm, xlabel=$x$, ylabel=$y$, align=center] \addplot table [mark=none, x index=0, y index=1, col sep=comma] {../data/btree_rand0.csv}; \end{axis} \end{tikzpicture} \\
\digraph[scale=0.2]{btree1} {
    graph ["rankdir"="LR","bgcolor"="transparent"];
    "306115458" ["color"="black","fontcolor"="black","fontname"="Helvetica","fontsize"="20","style"="setlinewidth(2)","label"="*"];
    "944427387" ["color"="black","fontcolor"="black","fontname"="Helvetica","fontsize"="20","style"="setlinewidth(2)","label"="*"];
    "521081105" ["color"="black","fontcolor"="black","fontname"="Helvetica","fontsize"="20","style"="setlinewidth(2)","label"="+"];
    "1907431275" ["color"="black","fontcolor"="black","fontname"="Helvetica","fontsize"="20","style"="setlinewidth(2)","label"="+"];
    "1637061418" ["color"="black","fontcolor"="black","fontname"="Helvetica","fontsize"="20","style"="setlinewidth(2)","label"="+"];
    "1686100174" ["color"="black","fontcolor"="black","fontname"="Helvetica","fontsize"="20","style"="setlinewidth(2)","label"="+"];
    "22671767" ["color"="black","fontcolor"="black","fontname"="Helvetica","fontsize"="20","style"="setlinewidth(2)","label"="+"];
    "2024453272" ["color"="black","fontcolor"="black","fontname"="Helvetica","fontsize"="20","style"="setlinewidth(2)","label"="0.108"];
    "x" ["color"="black","fontcolor"="black","fontname"="Helvetica","fontsize"="20","style"="setlinewidth(2)"];
    "1374026904" ["color"="black","fontcolor"="black","fontname"="Helvetica","fontsize"="20","style"="setlinewidth(2)","label"="*"];
    "536765369" ["color"="black","fontcolor"="black","fontname"="Helvetica","fontsize"="20","style"="setlinewidth(2)","label"="+"];
    "317071334" ["color"="black","fontcolor"="black","fontname"="Helvetica","fontsize"="20","style"="setlinewidth(2)","label"="+"];
    "1224347463" ["color"="black","fontcolor"="black","fontname"="Helvetica","fontsize"="20","style"="setlinewidth(2)","label"="+"];
    "1472465" ["color"="black","fontcolor"="black","fontname"="Helvetica","fontsize"="20","style"="setlinewidth(2)","label"="+"];
    "2129221032" ["color"="black","fontcolor"="black","fontname"="Helvetica","fontsize"="20","style"="setlinewidth(2)","label"="*"];
    "1831477404" ["color"="black","fontcolor"="black","fontname"="Helvetica","fontsize"="20","style"="setlinewidth(2)","label"="+"];
    "1580297332" ["color"="black","fontcolor"="black","fontname"="Helvetica","fontsize"="20","style"="setlinewidth(2)","label"="-"];
    "1966250569" ["color"="black","fontcolor"="black","fontname"="Helvetica","fontsize"="20","style"="setlinewidth(2)","label"="-"];
    "1125381564" ["color"="black","fontcolor"="black","fontname"="Helvetica","fontsize"="20","style"="setlinewidth(2)","label"="+"];
    "370440646" ["color"="black","fontcolor"="black","fontname"="Helvetica","fontsize"="20","style"="setlinewidth(2)","label"="*"];
    "2005733474" ["color"="black","fontcolor"="black","fontname"="Helvetica","fontsize"="20","style"="setlinewidth(2)","label"="-"];
    "511717113" ["color"="black","fontcolor"="black","fontname"="Helvetica","fontsize"="20","style"="setlinewidth(2)","label"="+"];
    "98394724" ["color"="black","fontcolor"="black","fontname"="Helvetica","fontsize"="20","style"="setlinewidth(2)","label"="-0.64"];
    "2085002312" ["color"="black","fontcolor"="black","fontname"="Helvetica","fontsize"="20","style"="setlinewidth(2)","label"="0.186"];
    "1791045777" ["color"="black","fontcolor"="black","fontname"="Helvetica","fontsize"="20","style"="setlinewidth(2)","label"="0.308"];
    "2130772866" ["color"="black","fontcolor"="black","fontname"="Helvetica","fontsize"="20","style"="setlinewidth(2)","label"="0.638"];
    "728739494" ["color"="black","fontcolor"="black","fontname"="Helvetica","fontsize"="20","style"="setlinewidth(2)","label"="0.119"];
    "1237550792" ["color"="black","fontcolor"="black","fontname"="Helvetica","fontsize"="20","style"="setlinewidth(2)","label"="0.518"];
    "1712943792" ["color"="black","fontcolor"="black","fontname"="Helvetica","fontsize"="20","style"="setlinewidth(2)","label"="1.0"];
    "842741472" ["color"="black","fontcolor"="black","fontname"="Helvetica","fontsize"="20","style"="setlinewidth(2)","label"="pow"];
    "1766505436" ["color"="black","fontcolor"="black","fontname"="Helvetica","fontsize"="20","style"="setlinewidth(2)","label"="2.930"];
    "1164440413" ["color"="black","fontcolor"="black","fontname"="Helvetica","fontsize"="20","style"="setlinewidth(2)","label"="-"];
    "1" ["color"="black","fontcolor"="black","fontname"="Helvetica","fontsize"="20","style"="setlinewidth(2)","label"="One"];
    "944427387" -> "306115458" ["color"="blue","arrowhead"="normal","style"="setlinewidth(2)"];
    "521081105" -> "944427387" ["color"="blue","arrowhead"="normal","style"="setlinewidth(2)"];
    "1907431275" -> "521081105" ["color"="blue","arrowhead"="normal","style"="setlinewidth(2)"];
    "1637061418" -> "1907431275" ["color"="blue","arrowhead"="normal","style"="setlinewidth(2)"];
    "1686100174" -> "1637061418" ["color"="blue","arrowhead"="normal","style"="setlinewidth(2)"];
    "22671767" -> "1686100174" ["color"="blue","arrowhead"="normal","style"="setlinewidth(2)"];
    "2024453272" -> "22671767" ["color"="blue","arrowhead"="normal","style"="setlinewidth(2)"];
    "x" -> "22671767" ["color"="red","arrowhead"="normal","style"="setlinewidth(2)"];
    "x" -> "1374026904" ["color"="red","arrowhead"="normal","style"="setlinewidth(2)"];
    "x" -> "317071334" ["color"="blue","arrowhead"="normal","style"="setlinewidth(2)"];
    "x" -> "317071334" ["color"="red","arrowhead"="normal","style"="setlinewidth(2)"];
    "x" -> "1224347463" ["color"="red","arrowhead"="normal","style"="setlinewidth(2)"];
    "x" -> "1831477404" ["color"="blue","arrowhead"="normal","style"="setlinewidth(2)"];
    "x" -> "1966250569" ["color"="black","arrowhead"="normal","style"="setlinewidth(2)"];
    "x" -> "1125381564" ["color"="blue","arrowhead"="normal","style"="setlinewidth(2)"];
    "x" -> "2005733474" ["color"="black","arrowhead"="normal","style"="setlinewidth(2)"];
    "1374026904" -> "536765369" ["color"="blue","arrowhead"="normal","style"="setlinewidth(2)"];
    "536765369" -> "1637061418" ["color"="red","arrowhead"="normal","style"="setlinewidth(2)"];
    "317071334" -> "536765369" ["color"="red","arrowhead"="normal","style"="setlinewidth(2)"];
    "1224347463" -> "1472465" ["color"="blue","arrowhead"="normal","style"="setlinewidth(2)"];
    "1472465" -> "2129221032" ["color"="blue","arrowhead"="normal","style"="setlinewidth(2)"];
    "2129221032" -> "1907431275" ["color"="red","arrowhead"="normal","style"="setlinewidth(2)"];
    "1831477404" -> "1580297332" ["color"="black","arrowhead"="normal","style"="setlinewidth(2)"];
    "1580297332" -> "1472465" ["color"="red","arrowhead"="normal","style"="setlinewidth(2)"];
    "1966250569" -> "1831477404" ["color"="red","arrowhead"="normal","style"="setlinewidth(2)"];
    "1125381564" -> "370440646" ["color"="blue","arrowhead"="normal","style"="setlinewidth(2)"];
    "370440646" -> "2129221032" ["color"="red","arrowhead"="normal","style"="setlinewidth(2)"];
    "2005733474" -> "511717113" ["color"="red","arrowhead"="normal","style"="setlinewidth(2)"];
    "511717113" -> "370440646" ["color"="red","arrowhead"="normal","style"="setlinewidth(2)"];
    "98394724" -> "1686100174" ["color"="red","arrowhead"="normal","style"="setlinewidth(2)"];
    "2085002312" -> "1374026904" ["color"="blue","arrowhead"="normal","style"="setlinewidth(2)"];
    "1791045777" -> "1224347463" ["color"="blue","arrowhead"="normal","style"="setlinewidth(2)"];
    "2130772866" -> "1125381564" ["color"="red","arrowhead"="normal","style"="setlinewidth(2)"];
    "728739494" -> "511717113" ["color"="blue","arrowhead"="normal","style"="setlinewidth(2)"];
    "1237550792" -> "521081105" ["color"="red","arrowhead"="normal","style"="setlinewidth(2)"];
    "1712943792" -> "944427387" ["color"="red","arrowhead"="normal","style"="setlinewidth(2)"];
    "842741472" -> "306115458" ["color"="red","arrowhead"="normal","style"="setlinewidth(2)"];
    "1766505436" -> "842741472" ["color"="blue","arrowhead"="normal","style"="setlinewidth(2)"];
    "1164440413" -> "842741472" ["color"="red","arrowhead"="normal","style"="setlinewidth(2)"];
    "1" -> "1164440413" ["color"="black","arrowhead"="normal","style"="setlinewidth(2)"];
} & \begin{tikzpicture} \begin{axis}[title={}, width=0.4\textwidth, height=5cm, xlabel=$x$, ylabel=$y$, align=center] \addplot table [mark=none, x index=0, y index=1, col sep=comma] {../data/btree_rand1.csv}; \end{axis} \end{tikzpicture} \\
\digraph[scale=0.2]{btree2} {
    graph ["rankdir"="LR","bgcolor"="transparent"];
    "1007412025" ["color"="black","fontcolor"="black","fontname"="Helvetica","fontsize"="20","style"="setlinewidth(2)","label"="*"];
    "2053591126" ["color"="black","fontcolor"="black","fontname"="Helvetica","fontsize"="20","style"="setlinewidth(2)","label"="*"];
    "231756373" ["color"="black","fontcolor"="black","fontname"="Helvetica","fontsize"="20","style"="setlinewidth(2)","label"="+"];
    "887750041" ["color"="black","fontcolor"="black","fontname"="Helvetica","fontsize"="20","style"="setlinewidth(2)","label"="+"];
    "1010953501" ["color"="black","fontcolor"="black","fontname"="Helvetica","fontsize"="20","style"="setlinewidth(2)","label"="+"];
    "1423561005" ["color"="black","fontcolor"="black","fontname"="Helvetica","fontsize"="20","style"="setlinewidth(2)","label"="+"];
    "943870983" ["color"="black","fontcolor"="black","fontname"="Helvetica","fontsize"="20","style"="setlinewidth(2)","label"="+"];
    "x" ["color"="black","fontcolor"="black","fontname"="Helvetica","fontsize"="20","style"="setlinewidth(2)"];
    "1084502906" ["color"="black","fontcolor"="black","fontname"="Helvetica","fontsize"="20","style"="setlinewidth(2)","label"="-"];
    "1881561036" ["color"="black","fontcolor"="black","fontname"="Helvetica","fontsize"="20","style"="setlinewidth(2)","label"="+"];
    "52908367" ["color"="black","fontcolor"="black","fontname"="Helvetica","fontsize"="20","style"="setlinewidth(2)","label"="*"];
    "1279271200" ["color"="black","fontcolor"="black","fontname"="Helvetica","fontsize"="20","style"="setlinewidth(2)","label"="-"];
    "587153993" ["color"="black","fontcolor"="black","fontname"="Helvetica","fontsize"="20","style"="setlinewidth(2)","label"="+"];
    "1613095350" ["color"="black","fontcolor"="black","fontname"="Helvetica","fontsize"="20","style"="setlinewidth(2)","label"="-"];
    "1971764991" ["color"="black","fontcolor"="black","fontname"="Helvetica","fontsize"="20","style"="setlinewidth(2)","label"="+"];
    "1276261147" ["color"="black","fontcolor"="black","fontname"="Helvetica","fontsize"="20","style"="setlinewidth(2)","label"="+"];
    "18242360" ["color"="black","fontcolor"="black","fontname"="Helvetica","fontsize"="20","style"="setlinewidth(2)","label"="*"];
    "1527953000" ["color"="black","fontcolor"="black","fontname"="Helvetica","fontsize"="20","style"="setlinewidth(2)","label"="-"];
    "135640095" ["color"="black","fontcolor"="black","fontname"="Helvetica","fontsize"="20","style"="setlinewidth(2)","label"="*"];
    "1632682988" ["color"="black","fontcolor"="black","fontname"="Helvetica","fontsize"="20","style"="setlinewidth(2)","label"="*"];
    "359922172" ["color"="black","fontcolor"="black","fontname"="Helvetica","fontsize"="20","style"="setlinewidth(2)","label"="+"];
    "1136419747" ["color"="black","fontcolor"="black","fontname"="Helvetica","fontsize"="20","style"="setlinewidth(2)","label"="0.375"];
    "1785507932" ["color"="black","fontcolor"="black","fontname"="Helvetica","fontsize"="20","style"="setlinewidth(2)","label"="0.791"];
    "757004314" ["color"="black","fontcolor"="black","fontname"="Helvetica","fontsize"="20","style"="setlinewidth(2)","label"="1.261"];
    "996796369" ["color"="black","fontcolor"="black","fontname"="Helvetica","fontsize"="20","style"="setlinewidth(2)","label"="-0.65"];
    "1430439149" ["color"="black","fontcolor"="black","fontname"="Helvetica","fontsize"="20","style"="setlinewidth(2)","label"="0.236"];
    "1153447573" ["color"="black","fontcolor"="black","fontname"="Helvetica","fontsize"="20","style"="setlinewidth(2)","label"="0.494"];
    "1786294176" ["color"="black","fontcolor"="black","fontname"="Helvetica","fontsize"="20","style"="setlinewidth(2)","label"="-0.23"];
    "2082351774" ["color"="black","fontcolor"="black","fontname"="Helvetica","fontsize"="20","style"="setlinewidth(2)","label"="1.0"];
    "1730704097" ["color"="black","fontcolor"="black","fontname"="Helvetica","fontsize"="20","style"="setlinewidth(2)","label"="pow"];
    "1062635358" ["color"="black","fontcolor"="black","fontname"="Helvetica","fontsize"="20","style"="setlinewidth(2)","label"="3.506"];
    "265321659" ["color"="black","fontcolor"="black","fontname"="Helvetica","fontsize"="20","style"="setlinewidth(2)","label"="-"];
    "1" ["color"="black","fontcolor"="black","fontname"="Helvetica","fontsize"="20","style"="setlinewidth(2)","label"="One"];
    "2053591126" -> "1007412025" ["color"="blue","arrowhead"="normal","style"="setlinewidth(2)"];
    "231756373" -> "2053591126" ["color"="blue","arrowhead"="normal","style"="setlinewidth(2)"];
    "887750041" -> "231756373" ["color"="blue","arrowhead"="normal","style"="setlinewidth(2)"];
    "1010953501" -> "887750041" ["color"="blue","arrowhead"="normal","style"="setlinewidth(2)"];
    "1423561005" -> "1010953501" ["color"="blue","arrowhead"="normal","style"="setlinewidth(2)"];
    "943870983" -> "1423561005" ["color"="blue","arrowhead"="normal","style"="setlinewidth(2)"];
    "x" -> "943870983" ["color"="blue","arrowhead"="normal","style"="setlinewidth(2)"];
    "x" -> "1084502906" ["color"="black","arrowhead"="normal","style"="setlinewidth(2)"];
    "x" -> "52908367" ["color"="blue","arrowhead"="normal","style"="setlinewidth(2)"];
    "x" -> "52908367" ["color"="red","arrowhead"="normal","style"="setlinewidth(2)"];
    "x" -> "1971764991" ["color"="blue","arrowhead"="normal","style"="setlinewidth(2)"];
    "x" -> "135640095" ["color"="blue","arrowhead"="normal","style"="setlinewidth(2)"];
    "x" -> "1632682988" ["color"="blue","arrowhead"="normal","style"="setlinewidth(2)"];
    "x" -> "1632682988" ["color"="red","arrowhead"="normal","style"="setlinewidth(2)"];
    "1084502906" -> "1881561036" ["color"="red","arrowhead"="normal","style"="setlinewidth(2)"];
    "1881561036" -> "1423561005" ["color"="red","arrowhead"="normal","style"="setlinewidth(2)"];
    "52908367" -> "1279271200" ["color"="black","arrowhead"="normal","style"="setlinewidth(2)"];
    "1279271200" -> "587153993" ["color"="red","arrowhead"="normal","style"="setlinewidth(2)"];
    "587153993" -> "1613095350" ["color"="black","arrowhead"="normal","style"="setlinewidth(2)"];
    "1613095350" -> "1010953501" ["color"="red","arrowhead"="normal","style"="setlinewidth(2)"];
    "1971764991" -> "1276261147" ["color"="blue","arrowhead"="normal","style"="setlinewidth(2)"];
    "1276261147" -> "18242360" ["color"="blue","arrowhead"="normal","style"="setlinewidth(2)"];
    "18242360" -> "1527953000" ["color"="black","arrowhead"="normal","style"="setlinewidth(2)"];
    "1527953000" -> "887750041" ["color"="red","arrowhead"="normal","style"="setlinewidth(2)"];
    "135640095" -> "1276261147" ["color"="red","arrowhead"="normal","style"="setlinewidth(2)"];
    "1632682988" -> "359922172" ["color"="blue","arrowhead"="normal","style"="setlinewidth(2)"];
    "359922172" -> "18242360" ["color"="red","arrowhead"="normal","style"="setlinewidth(2)"];
    "1136419747" -> "943870983" ["color"="red","arrowhead"="normal","style"="setlinewidth(2)"];
    "1785507932" -> "1881561036" ["color"="blue","arrowhead"="normal","style"="setlinewidth(2)"];
    "757004314" -> "587153993" ["color"="blue","arrowhead"="normal","style"="setlinewidth(2)"];
    "996796369" -> "1971764991" ["color"="red","arrowhead"="normal","style"="setlinewidth(2)"];
    "1430439149" -> "135640095" ["color"="red","arrowhead"="normal","style"="setlinewidth(2)"];
    "1153447573" -> "359922172" ["color"="red","arrowhead"="normal","style"="setlinewidth(2)"];
    "1786294176" -> "231756373" ["color"="red","arrowhead"="normal","style"="setlinewidth(2)"];
    "2082351774" -> "2053591126" ["color"="red","arrowhead"="normal","style"="setlinewidth(2)"];
    "1730704097" -> "1007412025" ["color"="red","arrowhead"="normal","style"="setlinewidth(2)"];
    "1062635358" -> "1730704097" ["color"="blue","arrowhead"="normal","style"="setlinewidth(2)"];
    "265321659" -> "1730704097" ["color"="red","arrowhead"="normal","style"="setlinewidth(2)"];
    "1" -> "265321659" ["color"="black","arrowhead"="normal","style"="setlinewidth(2)"];
} & \begin{tikzpicture} \begin{axis}[title={}, width=0.4\textwidth, height=5cm, xlabel=$x$, ylabel=$y$, align=center] \addplot table [mark=none, x index=0, y index=1, col sep=comma] {../data/btree_rand2.csv}; \end{axis} \end{tikzpicture}
\end{tabular}
\caption{\label{tab:btrees} Some DFGs generated by \autoref{eq:btree_gen} with accompanying 2D plots.}
\end{table}

%\begin{figure}
%\begin{tikzpicture} \begin{axis}[title={Average Errors per 1k Evaluations for Random Sampling Strategy}, width=\textwidth, height=5cm, xlabel=$\text{Error threshold}$, ylabel=$\text{Errors/eval}$, align=center] \addplot table [mark=none, x index=0, y index=1, col sep=comma] {../data/urand_eff0.csv}; \end{axis} \end{tikzpicture} \\
%\end{figure}
Below, is an excerpt from an implementation of momentum SGD in the Kotlin$\nabla$ DSL:
%
\begin{kotlinlisting}
val model = Vec(D30) { x pow (it + 1) } dot weights
var update = Vec(D30) { 0.0 }

batches.forEach { i, batch ->
  val batchInputs = arrayOf(xBatchIn to batch.first, label to batch.second)
  val batchLoss = (model - label).magnitude()(*batchInputs)
  val weightGrads = (d(batchLoss) / d(weights))(*newWeights)
  update = beta * update + (1 - beta) * weightGrads
  newWeights = newWeights - alpha * update
}
\end{kotlinlisting}

\begin{table}[H]
\begin{tabular}{l}
\begin{tikzpicture} \begin{axis}[title={Oracle vs. Regression Model}, width=\textwidth, height=0.4\textwidth, xlabel=$x$, ylabel=$y$, legend style={at={(0.5,0.03)},anchor=south}, align=center] \addplot table [mark=none, x=x, y=oracle, col sep=comma] {../data/oracle_vs_model.csv}; \addlegendentry{Oracle $(f)$} \addplot table [mark=none, x=x, y=model, col sep=comma] {../data/oracle_vs_model.csv}; \addlegendentry{Model $(\hat{f})$} \end{axis} \end{tikzpicture} \\
\begin{tikzpicture} \begin{axis}[title={True vs. Surrogate Loss}, width=\textwidth, height=0.4\textwidth, xlabel=$x$, ylabel=Loss, legend style={at={(0.5,0.97)},anchor=north}, align=center] \addplot table [mark=none, x=x, y=true, col sep=comma] {../data/true_vs_surrogate_loss.csv}; \addlegendentry{True Loss $(\mathcal{L})$} \addplot table [mark=none, x=x, y=surrogate, col sep=comma] {../data/true_vs_surrogate_loss.csv} [postaction={decorate, decoration={markings,mark=between positions 0.8 and 0.97 step 0.01 with {\arrow[red,line width=.5pt]{>};}}}]; \addlegendentry{Surrogate Loss $(\hat{\mathcal{L}})$} \end{axis} \end{tikzpicture}
\end{tabular}
\caption{\label{tab:oracle_vs_model} Above: Ground truth and trained model predictions for a single expression. Below: A single particle attacks the model by seeking higher error on the surrogate loss.}
\end{table}

Our adversary (\autoref{alg:surrogate_attack}) takes as input the trained regression model $\hat{f}$, a set of new input-output pairs from the ground truth expression, and resumes the original training procedure on $\hat{f}$ using the supplied datapoints for a fixed number of epochs, to produce a new model $\hat{f}'$. We use $\hat{f}'$ to construct a surrogate loss $\hat{\mathcal{L}}(x) = \big(\hat{f}(x) - \hat{f}'(x)\big)^2$, which can be maximized using \autoref{alg:diff_adversary}. Maximizing the surrogate loss allows us to construct adversarial examples without direct access to the oracle, an often impractical assumption in real world settings. For both adversarial testing and uniform sampling strategies, we compare the average number of violations detected per evaluation.

\begin{figure}[H]
\begin{tikzpicture}
\begin{axis}[title={Average efficiency at error threshold $\sigma$ standard deviations above MSE}, width=\textwidth, height=7cm, xlabel=$\sigma$, ylabel=Average errors detected per label, legend pos=north east, align=center, xtick distance=0.5]
\addplot[smooth, blue] table [mark=none, x=x, y=y1, col sep=comma] {../data/seff.csv};
\addlegendentry{Probabilistic Generator}
\addplot[smooth, red] table [mark=none, x=x, y=y2, col sep=comma] {../data/seff.csv};
\addlegendentry{Differential Adversary}
\addplot [smooth, name path=upper1,draw=none] table[x=x, y1=y1, y1_err=y1_err, y expr=\thisrow{y1}+\thisrow{y1_err}, col sep=comma] {../data/seff.csv};
\addplot [smooth, name path=lower1,draw=none] table[x=x, y1=y1, y1_err=y1_err, y expr=\thisrow{y1}-\thisrow{y1_err}, col sep=comma] {../data/seff.csv};
\addplot [fill=blue!10] fill between[of=upper1 and lower1];
\addplot [smooth, name path=upper2,draw=none] table[x=x, y2=y2, y2_err=y2_err, y expr=\thisrow{y2}+\thisrow{y2_err}, col sep=comma] {../data/seff.csv};
\addplot [smooth, name path=lower2,draw=none] table[x=x, y2=y2, y2_err=y2_err, y expr=\thisrow{y2}-\thisrow{y2_err}, col sep=comma] {../data/seff.csv};
\addplot [fill=red!10] fill between[of=upper2 and lower2];
\end{axis}
\end{tikzpicture}
\caption{By construction, our shrinker detects a greater number of errors per evaluation than one which does not take the gradient into consideration.}
\label{fig:pbt_comparison}
\end{figure}

Above, we show the number of violations exceeding error threshold $\sigma$ standard deviations above the mean squared error (MSE) on the true loss $\mathcal{L}(x) = \big(f(x) - \hat{f}(x)\big)^2$. On average, our adversary exhibits a $28\%$ improvement over the probabilistic baseline across all thresholds. We hypothesize that training the surrogate loss to convergence would further widen this margin, albeit potentially at the cost of generalization on other expressions.

% Furthermore, we hypothesize that if a sufficiently large fraction of the input space existed where $T$ were false, then as we sample from that space, the probability of detection would approach 1:
%
% \begin{equation}
%     (\forall i \in I^\dagger, p(i) \implies \neg T) \implies \lim_{|x|\to \infty}Prob(\hat{T}=False) = 1
% \end{equation}

% Suppose a machine learning department has a budget $B$ and fixed cost $C$ for labeling a single datapoint. Given an error threshold $e_{min}$ and access to $\frac{B}{C}$ new inputs from the oracle, suppose our method detects more errors than a uniform random sampling strategy over the input space. Conversely, to detect the same number of errors, we require a lower budget than a random sampling strategy.

\section{Conclusion}

In this chapter we have visited some interesting ideas for validating intelligent systems from the perspective of software engineering and machine learning. We have seen a curious resemblance between some new and old ideas in fuzz testing and adversarial learning. We have proposed new a framework for evaluating differentiable programs in a low-cost manner and shown our approach is more data-efficient than a random search strategy, employed by most automated testing frameworks. This enables us to detect a greater number of errors with a lower computational and data collection budget. The author wishes to thank Liam Paull for providing a number of very helpful suggestions and Stephen Samuel for the excellent \href{https://github.com/kotlintest/kotlintest}{KotlinTest}~\citep{kotlintest} library. This work was partly inspired by \citet{lample2019deep}, in particular the expression tree generator from ~\autoref{sec:prob_ad_test}.