\chapter*{Abstract}

Programming tools are computer programs which help humans program computers. Tools come in all shapes and forms, from editors and compilers to debuggers and profilers. Each of these tools facilitates a core task in the programming workflow which consumes cognitive resources when performed manually. In this thesis, we explore several tools that facilitate the process of building intelligent systems, and which reduce the cognitive effort required to design, develop, test and deploy intelligent software systems. First, we introduce an integrated development environment (IDE) for programming Robot Operating System (ROS) applications, called Hatchery (\autoref{ch:hatchery}). Second, we describe Kotlin$\nabla$, a language and type system for differentiable programming, an emerging paradigm in machine learning (\autoref{ch:kotlingrad}). Third, we propose a new algorithm for automatically testing differentiable programs, drawing inspiration from techniques in adversarial and metamorphic testing (\autoref{ch:difftest}), and demonstrate its empirical efficiency in the regression setting. Fourth, we explore a container infrastructure based on Docker, which enables reproducible deployment of ROS applications on the Duckietown platform (\autoref{ch:ducker}). Finally, we reflect on the current state of programming tools for these applications and speculate what intelligent systems programming might look like in the future (\autoref{ch:conclusion}).

\noindent\textbf{Keywords}: intelligent systems, machine learning, type systems, embedded systems, distributed systems, programming languages, functional programming, differentiable programming, probabilistic programming, programming tools, compilers, automatic differentiation, backpropagation, automated testing, fuzzing, metamorphic testing, property-based testing, generative modeling, static analysis, build automation, continuous integration, virtual machines, ROS, Kotlin, Docker, Duckietown.

\chapter*{R\'esum\'e}

\vspace{-40pt} Les outils de programmation sont des programmes informatiques qui aident les humains \`a programmer des ordinateurs. Les outils sont de toutes formes et tailles, par exemple les \'editeurs, les compilateurs, les d\'ebogueurs et les profileurs. Chacun de ces outils facilite une t\^ache principale dans le flux de travail de programmation qui consomme des ressources cognitives lorsqu'il est effectu\'e manuellement. Dans cette th\`ese, nous explorons plusieurs outils qui facilitent le processus de construction de syst\`emes intelligents et qui r\'eduisent l'effort cognitif requis pour concevoir, d\'evelopper, tester et d\'eployer des syst\`emes logiciels intelligents. Tout d'abord, nous introduisons un environnement de d\'eveloppement int\'egr\'e (EDI) pour la programmation d'applications Robot Operating System (ROS), appel\'e Hatchery (\autoref{ch:hatchery}). Deuxi\`emement, nous d\'ecrivons Kotlin$\nabla$, un syst\`eme de langage et de type pour la programmation diff\'erenciable, un paradigme \'emergent dans l'apprentissage automatique (\autoref{ch:kotlingrad}). Troisi\`emement, nous proposons un nouvel algorithme pour tester automatiquement les programmes diff\'erenciables, en nous inspirant des techniques de tests contradictoires et m\'etamorphiques (\autoref{ch:difftest}), et d\'emontrons son efficacit\'e empirique dans le cadre de la r\'egression. Quatri\`emement, nous explorons une infrastructure de conteneurs bas\'ee sur Docker, qui permet un d\'eploiement reproductible des applications ROS sur la plate-forme Duckietown (\autoref{ch:ducker}). Enfin, nous r\'efl\'echissons \`a l'\'etat actuel des outils de programmation pour ces applications et sp\'eculons \`a quoi pourrait ressembler la programmation de syst\`emes intelligents \`a l'avenir (\autoref{ch:conclusion}).

\noindent\textbf{Mots-cl\'es}: syst\`emes intelligents, apprentissage automatique, syst\`emes de types, syst\`emes embarqu\'es, syst\`emes distribu\'es, langages de programmation, programmation fonctionnelle, programmation diff\'erenciable, programmation probabiliste, outils de programmation, compilateurs, diff\'erenciation automatique, r\'etropropagation, test automatis\'e, fuzzing, test m\'etamorphique, test de propri\'et\'e, mod\'elisation g\'en\'erative, analyse statique, moteur de production, int\'egration continue, machines virtuelles, ROS, Kotlin, Docker, Duckietown.

\chapter*{Acknowledgements}

\vspace{-60pt} I would like to thank Gimmey, Mom, Uncle Mark, and Dad for their unfailing love and support. \href{http://hannelita.com/}{Hanneli Tavante} for teaching me type theory and the beauty of functional programming. Siyan Wang for his fellowship and adventures. \href{https://laverne.edu/directory/person/xiaoyan-liu/}{Xiaoyan Liu} for planting in me the seed of mathematics. Uncle Andy for watering the seed for many years. Aunt Shannon, Adam Devoe and Jacquie Kirrane for encouraging me to pursue grad school. \href{https://www.cs.rit.edu/~anh/}{Arthur Nunes-Harwitt} for teaching me algorithmic differentiation a long time ago. \href{https://www.sas.rochester.edu/bcs/people/faculty/miller_renee/index.html}{Renee Miller} for sparking my interest in neural science. \href{http://blog.locut.us}{Ian Clarke} for showing me a clever new language called \href{https://kotlinlang.org/}{Kotlin}. \href{https://hadihariri.com/}{Hadi Hariri} for putting more trust in me than I deserved. \href{https://www.jooq.org/}{Lukas Eder} and \href{https://jonnyzzz.com/}{Eugene Petrenko} for showing me the magic of type-safe DSLs and giving me advice about grad school. \href{https://github.com/rusi}{Rusi Hristov} for his patience and mentorship. \href{https://scholar.google.ca/citations?user=PsKlNzUAAAAJ}{Dmitry Serdyuk} and \href{http://kastnerkyle.github.io/}{Kyle Kastner} for introducing me to Montr\'eal and kindly welcoming me into the speech reading group. \href{https://scholar.google.ca/citations?user=-Ss9QGkAAAAJ}{Isabela Albuquerque} and \href{https://scholar.google.ca/citations?user=hkO47vsAAAAJ}{Jo\~ao Monteiro} for showing me what good research looks like. \href{https://takeitallsource.github.io}{Manfred Diaz} and \href{https://pointersgonewild.com/}{Maxime Chevalier Boisvert} for the inspiration, conversations and feedback. \href{https://fgolemo.github.io/}{Florian Golemo} for his excellent engineering and architectural advice. \href{http://TurnerComputing.com}{Ryan Turner}, \href{https://saikrishna-1996.github.io}{Saikrishna Gottipati}, \href{https://maivincent.github.io}{Vincent Mai}, \href{https://krrish94.github.io/}{Krishna Murthy}, \href{https://bhairavmehta95.github.io/}{Bhairav Mehta}, \href{https://mila.quebec/en/person/christos-tsirigotis/}{Christos Tsirigotis}, \href{http://www.solomatov.me/}{Konstantin Solomatov} and \href{https://www.seas.upenn.edu/~xsi/}{Xujie Si} for the interesting conversations. \href{https://scholar.google.ca/citations?user=bn4xHHIAAAAJ}{Pascal Lamblin}, \href{http://breuleux.net}{Olivier Breleux} and \href{https://scholar.google.ca/citations?user=XE9SDzgAAAAJ}{Bart van Merri\"enboer} and for lighting the path between ML and PL. \href{http://conal.net/}{Conal Elliott} for teaching me the importance of simplicity and denotational semantics. \href{http://christianperone.com}{Christian Perone} for introducing me to PyTorch, \href{https://research.jetbrains.org/researchers/altavir}{Alexander Nozik}, \href{https://twitter.com/headinthebox}{Erik Meijer}, \href{https://scholar.google.com/citations?user=IcuGXgcAAAAJ}{Kiran Gopinathan}, \href{https://medium.com/@elizarov}{Roman Elizarov}, \href{https://cquic.unm.edu/member/jacob.miller/}{Jacob Miller} and \href{http://www.adampocock.com/}{Adam Pocock} for their useful comments and feedback related to Kotlin$\nabla$. \href{https://miltos.allamanis.com/}{Miltos Allamanis} for showing me there is room for SE in ML. \href{https://diro.umontreal.ca/accueil/}{Celine Begin} at the Universit\'e de Montr\'eal for helping a stranger on a cold winter's eve in 2017. \href{https://www.iro.umontreal.ca/~monnier/}{Stefan Monnier} for thoughtfully and thoroughly replying to my rambling emails. \href{https://censi.science/}{Andrea Censi} for his advice and encouragement. Last but not least, I wish to thank my brilliant advisors \href{http://liampaull.ca/}{Liam Paull} for taking a chance on an out-of-distribution sample, providing strong gradients and giving me far more credit than I deserved, and \href{https://michalis.famelis.info/}{Michalis Famelis} for teaching me the value of intuitionistic logic, formal methods, and self-discipline. Thank you ever so much!