%\chapter*{Abstract}
%
%Programming tools are computer programs which help humans program computers. Tools come in all shapes and forms, from editors and compilers to debuggers and profilers. Each of these tools facilitates a core task in the programming workflow which consumes cognitive resources when performed manually. In this thesis, we explore several tools that facilitate the process of building intelligent systems, and which reduce the cognitive effort required to design, develop, test and deploy intelligent software systems. First, we introduce an integrated development environment (IDE) for programming Robot Operating System (ROS) applications, called Hatchery (\autoref{ch:hatchery}). Second, we describe Kotlin$\nabla$, a language and type system for differentiable programming, an emerging paradigm in machine learning (\autoref{ch:kotlingrad}). Third, we propose a new algorithm for automatically testing differentiable programs, drawing inspiration from techniques in adversarial and metamorphic testing (\autoref{ch:difftest}), and demonstrate its empirical efficiency in the regression setting. Fourth, we explore a container infrastructure based on Docker, which enables reproducible deployment of ROS applications on the Duckietown platform (\autoref{ch:ducker}). Finally, we reflect on the current state of programming tools for these applications and speculate what intelligent systems programming might look like in the future (\autoref{ch:conclusion}).
%
%\noindent\textbf{Keywords}: intelligent systems, machine learning, type systems, embedded systems, distributed systems, programming languages, functional programming, differentiable programming, probabilistic programming, programming tools, compilers, automatic differentiation, backpropagation, automated testing, fuzzing, metamorphic testing, property-based testing, generative modeling, static analysis, build automation, continuous integration, virtual machines, ROS, Kotlin, Docker, Duckietown.

\chapter*{R\'esum\'e}

\vspace{-40pt} Les outils de programmation sont des programmes informatiques qui aident les humains \`a programmer des ordinateurs. Les outils sont de toutes formes et tailles, par exemple les \'editeurs, les compilateurs, les d\'ebogueurs et les profileurs. Chacun de ces outils facilite une t\^ache principale dans le flux de travail de programmation qui consomme des ressources cognitives lorsqu'il est effectu\'e manuellement. Dans cette th\`ese, nous explorons plusieurs outils qui facilitent le processus de construction de syst\`emes intelligents et qui r\'eduisent l'effort cognitif requis pour concevoir, d\'evelopper, tester et d\'eployer des syst\`emes logiciels intelligents. Tout d'abord, nous introduisons un environnement de d\'eveloppement int\'egr\'e (EDI) pour la programmation d'applications Robot Operating System (ROS), appel\'e Hatchery (\autoref{ch:hatchery}). Deuxi\`emement, nous d\'ecrivons Kotlin$\nabla$, un syst\`eme de langage et de type pour la programmation diff\'erenciable, un paradigme \'emergent dans l'apprentissage automatique (\autoref{ch:kotlingrad}). Troisi\`emement, nous proposons un nouvel algorithme pour tester automatiquement les programmes diff\'erenciables, en nous inspirant des techniques de tests contradictoires et m\'etamorphiques (\autoref{ch:difftest}), et d\'emontrons son efficacit\'e empirique dans le cadre de la r\'egression. Quatri\`emement, nous explorons une infrastructure de conteneurs bas\'ee sur Docker, qui permet un d\'eploiement reproductible des applications ROS sur la plate-forme Duckietown (\autoref{ch:ducker}). Enfin, nous r\'efl\'echissons \`a l'\'etat actuel des outils de programmation pour ces applications et sp\'eculons \`a quoi pourrait ressembler la programmation de syst\`emes intelligents \`a l'avenir (\autoref{ch:conclusion}).

\noindent\textbf{Mots-cl\'es}: syst\`emes intelligents, apprentissage automatique, syst\`emes de types, syst\`emes embarqu\'es, syst\`emes distribu\'es, langages de programmation, programmation fonctionnelle, programmation diff\'erenciable, programmation probabiliste, outils de programmation, compilateurs, diff\'erenciation automatique, r\'etropropagation, test automatis\'e, fuzzing, test m\'etamorphique, test de propri\'et\'e, mod\'elisation g\'en\'erative, analyse statique, moteur de production, int\'egration continue, machines virtuelles, ROS, Kotlin, Docker, Duckietown.

\chapter*{Acknowledgements}

\vspace{-60pt} Je voudrais remercier Gimmey, maman, oncle Mark et papa pour leur amour et leur soutien sans faille. \href{http://hannelita.com/}{Hanneli Tavante} pour m'avoir enseigné la théorie des types et la beauté de la programmation fonctionnelle. \href{https://laverne.edu/directory/person/xiaoyan-liu/}{Xiaoyan Liu} pour avoir semé en moi la graine des mathématiques. Oncle Andy pour avoir arrosé la graine pendant de nombreuses années. Tante Shannon, Adam Devoe et Jacquie Kirrane pour m'avoir encouragé à poursuivre des études supérieures. \href{https://www.cs.rit.edu/~anh/}{Arthur Nunes-Harwitt} pour m'avoir enseigné la différenciation algorithmique il y a longtemps. \href{https://www.sas.rochester.edu/bcs/people/faculty/miller_renee/index.html}{Renee Miller} pour avoir suscité mon intérêt pour la science neuronale. \href{http://blog.locut.us}{Ian Clarke} pour m'avoir montré un nouveau langage intelligent appelé \href{https://kotlinlang.org/}{Kotlin}. \href{https://hadihariri.com/}{Hadi Hariri} pour me faire plus confiance que je ne le méritais. \href{https://www.jooq.org/}{Lukas Eder} et \href{https://jonnyzzz.com/}{Eugene Petrenko} pour m'avoir montré la magie des DSL sécurisées et m'avoir donné des conseils sur études supérieures. \href{https://github.com/rusi}{Rusi Hristov} pour sa patience et son mentorat. \href{https://scholar.google.ca/citations?user=PsKlNzUAAAAJ}{Dmitry Serdyuk} et \href{http://kastnerkyle.github.io/}{Kyle Kastner} pour m'avoir présenté à Montr \ 'eal et me souhaitant la bienvenue dans le groupe de lecture de discours. \href{https://scholar.google.ca/citations?user=-Ss9QGkAAAAJ}{Isabela Albuquerque} et \href{https://scholar.google.ca/citations?user=hkO47vsAAAAJ}{Jo\~ao Monteiro} pour m'avoir montré à quoi ressemblent de bonnes recherches. \href{https://takeitallsource.github.io}{Manfred Diaz} et \href{https://pointersgonewild.com/}{Maxime Chevalier Boisvert} pour l'inspiration, les conversations et les commentaires. \href{https://fgolemo.github.io/}{Florian Golemo} pour ses excellents conseils d'ingénierie et d'architecture. \href{http://TurnerComputing.com}{Ryan Turner}, \href{https://saikrishna-1996.github.io}{Saikrishna Gottipati}, \href{https://maivincent.github.io}{Vincent Mai}, \href{https://krrish94.github.io/}{Krishna Murthy}, \href{https://bhairavmehta95.github.io/}{Bhairav ​​Mehta}, \href{https://mila.quebec/fr/person/christos-tsirigotis/}{Christos Tsirigotis}, \href{http://www.solomatov.me/}{Konstantin Solomatov} et \href{https://www.seas.upenn.edu/~ xsi/}{Xujie Si} pour les conversations intéressantes. \href{https://scholar.google.ca/citations?user=bn4xHHIAAAAJ}{Pascal Lamblin}, \href{http://breuleux.net}{Olivier Breleux} et \href{https://scholar.google.ca/citations? user = XE9SDzgAAAAJ}{Bart van Merri\"enboer} et pour éclairer le chemin entre ML et PL. \href{http://conal.net/}{Conal Elliott} pour m'apprendre l'importance de la simplicité et la sémantique dénotationnelle. \href{http://christianperone.com}{Christian Perone} pour m'avoir présenté PyTorch, \href{https://research.jetbrains.org/researchers/altavir}{Alexander Nozik}, \href{https://twitter.com/headinthebox}{Erik Meijer}, \href{https://scholar.google.com/citations?user=IcuGXgcAAAAJ}{Kiran Gopinathan}, \href{https://medium.com/@elizarov}{Roman Elizarov}, \href{https://cquic.unm.edu/member/jacob.miller/}{Jacob Miller} et \href{http://www.adampocock.com/}{Adam Pocock} pour leurs commentaires et retours utiles concernant Kotlin$\nabla$. \href{https://miltos.allamanis.com/}{Miltos Allamanis} pour m'avoir montré qu'il y a de la place pour SE en ML. \href{https://diro.umontreal.ca/accueil/}{Celine Begin} à l'Universit\'e de Montr\'eal pour avoir aidé un étranger la veille d'un hiver froid en 2017. \href{https://www.iro.umontreal.ca/~monnier/}{Stefan Monnier} pour avoir répondu de manière réfléchie et approfondie à mes courriels errants. \href{https://censi.science/}{Andrea Censi} pour ses conseils et ses encouragements. Enfin et surtout, je tiens à remercier mes brillants conseillers \href{http://liampaull.ca/}{Liam Paull} d'avoir pris une chance sur un échantillon hors distribution, en fournissant de forts gradients et en me donnant beaucoup plus de crédit que je ne le méritais, et \href{https://michalis.famelis.info/}{Michalis Famelis} pour m'avoir enseigné la valeur de la logique intuitionniste, des méthodes formelles et de l'autodiscipline. Merci infiniment!
